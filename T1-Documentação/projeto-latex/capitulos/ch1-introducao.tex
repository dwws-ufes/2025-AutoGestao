% ==============================================================================
% Projeto de Sistema - Nome do Aluno
% Capítulo 1 - Introdução
% ==============================================================================
\chapter{Introdução}
\label{sec-intro}
\vspace{-1cm}

Este documento apresenta o projeto (\textit{design}) do sistema \emph{\imprimirtitulo}. Neste sistema, será possível realizar a gestão interna de venda de veículos de uma empresa. Como resultado, espera-se fazer o controle do portfólio de veículos que a loja possui para venda ou que a venda já foi realizada. Por meio do controle de estados dos veículos, será possível que os vendedores apontem e visualizem que um veículo já está em negociação, evitando desconforto entre vendedores e clientes.

Ainda, o sistema será capaz de manter o registro dos funcionários para controle de suas comissões a partir das vendas realizadas. Assim, ao vender um veículo e registrar o faturamento de sua venda, será contabilizada sua comissão para recebimento em determinado período. Por fim, a ideia é que clientes que comprem veículos não sejam os usuários do sistema, ou seja, não serão realizadas compras pelo sistema, ele será para gestão interna da empresa.

Além desta introdução, este documento está organizado da seguinte forma: 
a Seção~\ref{sec-plataforma} apresenta a plataforma de software utilizada na implementação do sistema;
a Seção~\ref{sec-arquitetura} apresenta a arquitetura de software; por fim, 
a Seção~\ref{sec-frameweb} apresenta os modelos FrameWeb que descrevem os componentes da arquitetura.

